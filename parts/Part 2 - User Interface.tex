% Part 2 - User Interface.tex

\pagebreak
\hypertarget{UserInterface}{}
\section{Navigating the User Interface}

ChopChop's UI design allows users to view all of the information you need in three main tabs. However, that might lead to a slightly steeper learning curve. Hence, this section aims to give you a breakdown of the GUI's various components.

Specifically, this section covers:
\begin{numberedlist}
	& \hyperlink{CommandBox}{Command Box}
	& \hyperlink{CommandOutput}{Command Output}
	& \hyperlink{RecipeButton}{Recipe Button}
	& \hyperlink{IngredientButton}{Ingredient Button}
	& \hyperlink{RecommendationButton}{Recommendation Button}
	& \hyperlink{StatisticsBox}{Statistics Box}
	& \hyperlink{RecipeTiles}{Recipe Tile}
	& \hyperlink{IngredientTiles}{Ingredient Tile}
	& \hyperlink{RecipeView}{Recipe View}
\end{numberedlist}


First, here is the overview of the components of the recipe list view in ChopChop, which you will be greeted with when first
starting the application:

\begin{figure}[!htbp]\centering\ContinuedFloat*
	\img{145mm}{images/recipe_panel_description.png}
	\caption{The Recipe List Panel of ChopChop}
	\label{fig:recipe-list}
\end{figure}


\pagebreak
\hypertarget{CommandBox}{}
\subsection{Command Box}
	ChopChop does your bidding by listening to your commands — the \emph{Command Box} is where you type your textual commands.
	After typing your commands, press \kbd{enter} to input the command. You can also use the \kbd{up} and \kbd{down} arrow keys
	to navigate through the history of commands you've typed previously.

	To learn about the commands you can use, check out our \hyperlink{CommandSummary}{command summary} for a quick overview or
	our \hyperlink{Commands}{commands} for a detailed list.

	If you have yet to check out ChopChop's \hyperlink{TabCompletion}{tab completion} section, do drop by to learn this handy feature!
% end subsection


\hypertarget{CommandOutput}{}
\subsection{Command Output}
	ChopChop will always display textual responses to the commands you input — the \emph{Command Output} is where you can view
	the responses.
% end subsection


\hypertarget{RecipeButton}{}
\subsection{Recipe Button}
	ChopChop is able to display all of your recipes as \hyperlink{RecipeTiles}{Recipe Tiles} in the \emph{Main Display Area} as shown in
	\autoref{fig:recipe-list} when you press the \emph{Recipes} button. The 4 buttons at the bottom of our GUI will take on a darker
	shade when they are currently selected.
% end subsection


\hypertarget{IngredientButton}{}
\subsection{Ingredient Button}
	ChopChop is able to display all of your ingredients as \hyperlink{IngredientTiles}{Ingredient Tiles} in the \emph{Main Display Area} as
	shown in \autoref{fig:ingredient-list} when you press the \emph{Ingredients} button.
% end subsection


\hypertarget{RecommendationButton}{}
\subsection{Recommendation Button}
	ChopChop is able to recommend you recipes to cook if you want it to! The recipes that are recommended will be displayed as
	\hyperlink{RecipeTiles}{Recipe Tiles} in the \emph{Main Display Area} when you press the \emph{Recommendations} button.
% end subsection


\hypertarget{StatisticsBox}{}
\subsection{Statistics Box}
	ChopChop is able to produce statistics based on the food you make! To view your ingredient or recipe usage statistics, simply
	input into the \emph{Command Box} one of the \hyperlink{GroupStatsCommands}{\mono{stats commands}} available.
% end subsection


\hypertarget{RecipeTiles}{}
\subsection{Recipe Tiles}
	ChopChop is able to show you your recipes in the form of tiles — simply click on a tile to view the detailed information about
	the selected recipe in the \emph{Main Display Area}, as shown in \autoref{fig:recipe-view}.
% end subsection


\hypertarget{IngredientTiles}{}
\subsection{Ingredient Tiles}
	ChopChop is able to show you your ingredients in the form of tiles, as shown in \autoref{fig:ingredient-list}.
	In each tile, there are 5 components:

	\begin{numberedlist}
		& The \emph{Index} shows the index of the ingredient for easier referencing
		& The \emph{Expiry Date} of the ingredient
		& The \emph{Quantity} of the ingredient
		& The \emph{Ingredient Name} of the ingredient
		& The \emph{Tags} associated with the ingredient
	\end{numberedlist}

	Note: if the tags are too long, some graphical glitches might occur. This will be fixed in the next version!

	\begin{figure}[!htbp]\centering\ContinuedFloat
		\img{145mm}{images/ingredient_panel_description.png}
		\caption{The Ingredient List Panel of ChopChop}
		\label{fig:ingredient-list}
	\end{figure}

	When showing the expiry date, ChopChop only displays the earliest expiry date out of all the ingredient *sets* that exist.
	For example, if you have \emph{\SI{500}{\milli\litre}} of milk expiring on the 11th of November and \emph{\SI{1.5}{\litre}}
	expiring on the 19th, the expiry date will be shown as the 11th.

% end subsection


\pagebreak

\hypertarget{RecipeView}{}
\subsection{Recipe View}

	When clicking on a recipe or \hyperlink{ViewRecipeCommand}{\mono{view}}-ing it, ChopChop will show the full details of the recipe.
	In this view, you can see the entirety of the recipe:

	\begin{numberedlist}
		& The \emph{Name} of the recipe
		& The \emph{Tags} associated with the recipe
		& The \emph{Ingredients} used by the recipe (along with their quantities)
		& The \emph{Steps} involved in making the recipe
	\end{numberedlist}

	\begin{figure}[!htbp]\centering\ContinuedFloat
		\img{145mm}{images/recipe_view_description.png}
		\caption{The Recipe View Panel of ChopChop}
		\label{fig:recipe-view}
	\end{figure}

% end subsection
% end section
