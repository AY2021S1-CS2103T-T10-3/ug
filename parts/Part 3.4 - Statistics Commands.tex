% Part 3.4 - Statistics Commands.tex

\pagebreak
\hypertarget{GroupStatsCommands}{}
\subsection{Statistics Commands (Travis)}

Whenever you make a recipe or consume an ingredient, ChopChop makes a record, which you can view with the commands below.
The output of these statistics commands are shown in the \hyperlink{StatisticsBox}{statistics box} on the right side of
ChopChop's interface.


\hypertarget{StatsRecipeTopCommand}{}
\subsubsection{Listing Top Recipes — \texttt{\bld{stats} recipe top}}

	This command shows a list of recipes that were made the most, in descending order (the recipe listed first was made the most
	number of times).

	Note that, even if a recipe was deleted (with \hyperlink{DeleteRecipeCommand}{\mono{delete recipe}}, its cooking records will still
	exist in ChopChop. To remove these statistics, you can use \hyperlink{StatsRecipeClearCommand}{\mono{stats recipe clear}} to clear
	them for all recipes.

	\bld{Usage:} \begin{blockofcode}{text}
		stats recipe top
	\end{blockofcode}

	For example, here is the output for someone who really really loves pancakes:

	\begin{figure}[!htbp]\centering
		\img{60mm}{images/stats_recipe_top.png}
		\caption{No pancakes were sacrificed for this picture}
	\end{figure}

% end subsubsection





\pagebreak
\hypertarget{StatsRecipeRecentCommand}{}
\subsubsection{Listing Recent Recipes — \texttt{\bld{stats} recipe recent}}

	This command shows a list of the 10 most recently-made recipes, with the most recent one at the top of the list. As with the other
	statistics commands, deleting a recipe from ChopChop does not delete it from this list.

	Note that this is also the view that is displayed by default when no statistics commands have been used yet.

	\bld{Usage:} \begin{blockofcode}{text}
		stats recipe recent
	\end{blockofcode}

	\begin{figure}[!htbp]\centering
		\img{60mm}{images/stats_recipe_recent.png}
		\caption{The recently-made recipes view}
	\end{figure}

% end subsubsection



\hypertarget{StatsRecipeMadeCommand}{}
\subsubsection{Listing Recipes within a Time Frame — \texttt{\bld{stats} recipe made}}

	This command shows a list of recipes that were made within the given time frame, arranged in descending chronological order.
	You can specify the lower bound (earliest date/time), upper bound (latest date/time), both, or neither.

	\bld{Usage:} \begin{blockofcode}{text}
		stats recipe made
			[/after <date-time>]
			[/before <date-time>]
	\end{blockofcode}

	The format of \mono{<date-time>} is as such: \mono{yyyy-mm-dd [hh:mm]}, where \mono{yyyy-mm-dd} is the
	\hyperlink{DateFormats}{familiar date format}, and \mono{hh:mm} is the optional time in 24-hour format (eg. \mono{18:30} for 6.30pm).

	If the time component is omitted, it is assumed to be midnight of the given day. If both \mono{/before} and \mono{/after} are omitted,
	then ChopChop shows recipes made on the current day, ie. from midnight to 23:59.

	Examples:
	\begin{bulletlist}
		& \mono{stats recipe made} \\
		  This shows all recipes made since the beginning of the current day

		& \mono{stats recipe made /after 2020-04-15} \\
		  This shows all recipes made after the 15th of April, 2020

		& \mono{stats recipe made /before 2020-12-25} \\
		  This shows all recipes made before the 25th of December, 2020

		& \mono{stats recipe made /after 2020-04-15 /before 2020-12-25} \\
		  This shows all recipes made between the 15th of April and 25th of December

		& \mono{stats recipe made /after 2020-01-01 08:00 /before 2020-01-01 23:00} \\
		  This shows all recipes made between 8am and 11pm on the 1st of January
	\end{bulletlist}

	For example, \mono{stats recipe made /after 2020-11-02 /before 2020-11-04 23:00} shows this output:

	\begin{figure}[!htbp]\centering\ContinuedFloat*
		\img{60mm}{images/stats_recipe_made_1.png}
		\caption{Showing the recipes made between a date range}
	\end{figure}

	\pagebreak
	Meanwhile, just executing \mono{stats recipe made} only shows the recipes made on the current day:

	\begin{figure}[!htbp]\centering\ContinuedFloat
		\img{60mm}{images/stats_recipe_made_2.png}
		\caption{Showing the recipes made on the current day}
	\end{figure}

% end subsubsection









\hypertarget{StatsRecipeClearCommand}{}
\subsubsection{Clearing Recipe History — \texttt{\bld{stats} recipe clear}}

	This command clears the history of the recipes that you've made from ChopChop. If you did this accidentally, don't worry,
	because you can \hyperlink{UndoCommand}{undo} this.

	\bld{Usage:} \begin{blockofcode}{text}
		stats recipe clear
	\end{blockofcode}

% end subsubsection



\pagebreak
\hypertarget{StatsIngredientRecentCommand}{}
\subsubsection{Listing Recent Ingredients — \texttt{\bld{stats} ingredient recent}}

	This command shows a list of the 10 most recently-used ingredients consumed by \hyperlink{MakeRecipeCommand}{\mono{make}}-ing recipes.
	It is similar to the \hyperlink{StatsRecipeRecentCommand}{\mono{stats recipe recent}} command as discussed above, except it deals
	with ingredients.

	\bld{Usage:} \begin{blockofcode}{text}
		stats ingredient recent
	\end{blockofcode}

	For example, after having just made \emph{Scrambled Eggs}, these will be the recently-used ingredients:

	\begin{figure}[!htbp]\centering
		\img{60mm}{images/stats_ingredient_recent.png}
		\caption{The recently used ingredients view}
	\end{figure}

% end subsubsection



\hypertarget{StatsIngredientUsedCommand}{}
\subsubsection{Listing Ingredients within a Time Frame — \texttt{\bld{stats} ingredient used}}

	This command shows a list of ingredients that were used within the given time frame, arranged in descending chronological order. You can specify the lower bound (earliest date/time), upper bound (latest date/time), both, or neither.

	This command behaves similarly to \hyperlink{StatsRecipeMadeCommand}{\mono{stats recipe made}} as discussed above, except that
	\mono{made} is replaced with \mono{used} instead. Otherwise, the arguments (\mono{/before}, \mono{/after}) function identically.

	\bld{Usage:} \begin{blockofcode}{text}
		stats ingredient used
			[/after <date-time>]
			[/before <date-time>]
	\end{blockofcode}

% end subsubsection


\hypertarget{StatsIngredientClearCommand}{}
\subsubsection{Clearing Ingredient History — \texttt{\bld{stats} ingredient clear}}

	This command clears the history of the ingredients that you've used in ChopChop. If you did this accidentally, don't worry, because you can \hyperlink{UndoCommand}{undo} this.

	\bld{Usage:} \begin{blockofcode}{text}
		stats ingredient clear
	\end{blockofcode}

% end subsubsection





% end subsection
