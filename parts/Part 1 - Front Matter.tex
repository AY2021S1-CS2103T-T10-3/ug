% Part 1 - Front Matter.tex

\pagebreak
\hypertarget{Introduction}{}
\section{Introduction}

ChopChop is a food recipe management system, which aims to help users manage their recipes and ingredients in an easy and effective
manner. It is a *desktop app*, optimised for use through typing textual commands; for fast typists, ChopChop will be able to manage
your recipes more efficiently than other applications. Even so, it also features a graphical user interface (GUI) to display
ingredients and recipes in an interactive manner.

Furthermore, our command interface also features \hyperlink{TabCompletion}{tab completion}, which will greatly increase the speed at
which you can use ChopChop by reducing the amount of typing required.

\subsection{Navigating this Document}
	This user guide provides an in-depth guide about how to use ChopChop; you can click on any section or command in the
	Table of Contents above to find answers or get step-by-step instructions. In addition, the \hyperlink{QuickStart}{quick start guide}
	provides an end-to-end setup process to help you get started with installing ChopChop.

	Specifically, this document covers:
	\begin{numberedlist}
		& The components of the user interface
		& The syntax and behaviour of the commands
		& Detailed usage examples, with step-by-step illustrated walkthroughs
		& Other usage notes and features
	\end{numberedlist}
% end subsection

\subsection{Document Conventions}

	In this document, some elements are styled differently for emphasis; these elements are:
	\begin{bulletlist}
		& \bulb{}    \tabto{20mm}--- indicating a useful piece of information that can make using ChopChop easier
		& \info{}    \tabto{20mm}--- indicating something that you should pay attention to
		& \mono{code}\tabto{20mm}--- indicating text that you can type into the \hyperlink{CommandBox}{command box}
		& \kbd{enter}\tabto{20mm}--- indicating keys that you press on your keyboard
		& \hyperlink{asdf}{link}\tabto{20mm}--- indicating clickable links
		& \href{https://example.com}{external link}\tabto{20mm}--- indicating clickable links, leading to external websites
	\end{bulletlist}
% end subsection
% end section


\pagebreak
\hypertarget{QuickStart}{}
\section{Quick Start}

To start using and experimenting with ChopChop, here are the steps you can follow:
\begin{numberedlist}
	& Ensure you have Java \bld{11} or above installed on your computer; you can obtain it from
		\href{https://adoptopenjdk.net}{AdoptOpenJDK}.
	& Download the latest \bld{chopchop.jar} from our \href{https://github.com/AY2021S1-CS2103T-T10-3/tp/releases}{Github Page}.
	& Copy the file to the folder you want to use as the \emph{home folder} for your ChopChop.
	& Double-click the file to start the app. The GUI similar to the below should appear in a few seconds. The app starts with some
		sample data for you to experiment with.
	\begin{figure}[!htbp]\centering
		\img{145mm}{images/sample_data.png}
		\caption{The initial state of ChopChop, including sample data}
	\end{figure}
\end{numberedlist}

Now that you have ChopChop installed, you can start to play around with the sample data — add recipes, ingredients, and more!
Some commands you could try include:

\begin{bulletlist}
	& \mono{add ingredient milk /qty 500ml /expiry 2020-11-09} \\
		This command adds the ingredient \enquote{milk}.
	& \mono{add recipe milkshake /ingredient milk /qty 250ml /step Add milk /step Shake} \\
		This command adds a recipe for \enquote{milkshake}.
\end{bulletlist}

To exit ChopChop, you can either use the \mono{quit} command, or simply close the application window.
% end section



\hypertarget{Overview}{}
\section{Overview}

ChopChop manages two key components --- ingredients and recipes --- and they will be the main pieces you will interact with.
Common to both are names and tags, the latter of which allow you to quickly group related ingredients or recipes together,
or to organise them in any way you desire.

Names and tags are both case insensitive, so \emph{pAnCaKeS} and \emph{Pancakes} refer to the same recipe. Note that you cannot
have duplicate recipes nor ingredients in ChopChop; items are duplicates if their names are the same.

\hypertarget{OverviewRecipes}{}
\subsection{Recipes}
	A recipe consists of a list of ingredients and associated quantities used, as well as a list of ordered steps in textual form.
% end subsection

\hypertarget{OverviewIngredients}{}
\subsection{Ingredients}
	An ingredient consists of a quantity with an associated unit, and an optional expiry date. Each ingredient can have multiple
	\emph{sets}, where each set is a given quantity of that ingredient, expiring on a certain date.

	For example, you might have \emph{\SI{500}{\milli\litre}} of milk that you bought last week that expires tomorrow, while you have
	another \emph{\SI{1.5}{\litre}} of milk that you bought today, expiring two weeks from now. ChopChop will track both these
	\emph{sets}, and will intelligently use the earliest-expiring set when doing its accounting.

	For a more in-depth look at how ChopChop handles quantities, see \hyperlink{QuantitiesAndUnits}{this section}.
% end subsection


\hypertarget{OverviewNameConstraints}{}
\subsection{Name Constraints}

	Recipe and ingredient names have the same constraints, of which there is only one — they cannot consist \emph{only} of a
	\mono{\#} followed by digits, to diambiguate them from numbered references (for the tech savvy, the regular expression is
	\mono{\#[0-9]+}).

	Here are some examples of names that are valid and invalid:

	\begin{nicetable}[1.3][0.4\textwidth]{ X[c,m] | X[c,m] }
		Name            & Valid             \\ \hline
		\mono{\#1}      & \emojicross{}     \\
		\mono{\#1234}   & \emojicross{}     \\
		\mono{\#asdf}   & \emojitick{}      \\
		\mono{\#1a}     & \emojitick{}      \\
		\mono{\#1 abc}  & \emojitick{}      \\
		\mono{\#12 34}  & \emojitick{}      \\
	\end{nicetable}
% end subsection
% end section

