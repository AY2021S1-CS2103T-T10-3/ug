% Part 3.1 - General Commands.tex

\pagebreak
\hypertarget{GroupGeneralCommands}{}
\subsection{General Commands}

These commands are general commands that do not operate on ingredients or recipes.

\hypertarget{HelpCommand}{}
\subsubsection{Getting Help — \texttt{\bld{help}}}

	This command shows a message with a link to this user guide; you can use it to easily access this page from the application.
	The link can be clicked, and will open this page in your web browser.

	Furthermore, you can also use this command to get help for specific commands. In that case, clicking the link will bring you
	to the corresponding section in the User Guide for that command.

	\bld{Usage:} \begin{blockofcode}{text}
		help [<command-name> [<command-target>]]
	\end{blockofcode}

	Examples:
	\begin{bulletlist}
	& If you simply want a link to this page, you can just use \mono{help}:
		\begin{figure}[!htbp]\centering\ContinuedFloat*
			\img{75mm}{images/help_message_1.png}
			\caption{The help message, with a link to this User Guide}
		\end{figure}

	& If you want help for a specific command --- for example, how to add recipes --- you can instead use \mono{help add recipe},
		which will give you a link that brings you to that specific section:
		\begin{figure}[!htbp]\centering\ContinuedFloat
			\img{75mm}{images/help_message_2.png}
			\caption{The help message for a specific command}
		\end{figure}

	\end{bulletlist}

% end subsubsection


\hypertarget{QuitCommand}{}
\subsubsection{Quitting ChopChop — \texttt{\bld{quit}}}
	This command quits ChopChop. You can rest assured that your data is automatically saved whenever a command is executed, so you
	do not need to save it manually before quitting.

	\bld{Usage:} \begin{blockofcode}{text}
		quit
	\end{blockofcode}
% end subsubsection




\hypertarget{UndoCommand}{}
\subsubsection{Undoing Commands — \texttt{\bld{undo}}}

	This command undoes the last undoable command. Undoable commands are commands that involve changes to recipes and ingredients stored in ChopChop. You can check the \hyperlink{CommandSummary}{command summary} for a quick list of which commands can be undone.

	\bld{Usage:} \begin{blockofcode}{text}
		undo
	\end{blockofcode}

	The most common scenario to undo a command is when accidentally deleting something; for example, suppose you wanted to delete the
	\emph{Salted Water} recipe, but you accidentally deleted the \emph{Scrambled Eggs} recipe instead:

	\begin{figure}[!htbp]\centering\ContinuedFloat*
		\img{145mm}{images/undo_1.png}
		\caption{Accidentally typing \enquote{Scrambled Eggs} instead of \enquote{Salted Water}}
	\end{figure}

	% asdf.
	\pagebreak
	Fear not, because fixing the mistake is a simple \mono{undo} away:
	\vspace{-0.5em}

	\begin{figure}[!htbp]\centering\ContinuedFloat
		\img{145mm}{images/undo_2.png}
		\caption{Simply use the \mono{undo} command}
	\end{figure}
	\vspace{-1.5em}

	After pressing \kbd{enter}, notice that the \emph{Scrambled Eggs} recipe is back:
	\vspace{-0.5em}

	\begin{figure}[!htbp]\centering\ContinuedFloat
		\img{145mm}{images/undo_3.png}
		\caption{The scrambled eggs were saved}
	\end{figure}
	\vspace{-2em} % double asdf.

% end subsubsection

\pagebreak
\hypertarget{RedoCommand}{}
\subsubsection{Redoing Commands — \texttt{\bld{redo}}}
	This command redoes the last redoable command, effectively functioning as an undo for undo itself. All undoable commands
	(as described \hyperlink{UndoCommand}{above}) can be redone.

	\bld{Usage:} \begin{blockofcode}{text}
		redo
	\end{blockofcode}

	For example, let's say that you changed your mind, and you didn't really want those scrambled eggs anyway; you can use \mono{redo}
	to redo the deletion of that recipe:

	\begin{figure}[!htbp]\centering\ContinuedFloat*
		\img{145mm}{images/redo_1.png}
		\caption{The scrambled eggs will be deleted again}
	\end{figure}

	\pagebreak
	After pressing \kbd{enter}, the scrambled eggs are now deleted:

	\begin{figure}[!htbp]\centering\ContinuedFloat
		\img{145mm}{images/redo_2.png}
		\caption{The scrambled eggs are now gone}
	\end{figure}


% end subsubsection



\hypertarget{ListRecommendationCommand}{}
\subsubsection{Listing Recommendations — \texttt{\bld{list} recommendations}}

	This command allows you to switch to the recommendations pane, to view recipe recommendations. There are currently two types
	of recommendations that ChopChop will make:

	\begin{romanlist}
		& Recipes that only use ingredients you have in stock
		& Recipes that use ingredients that are about to expire
	\end{romanlist}

	In the second case, only ingredients that expire within the next week will be taken into account.

	\bld{Usage:} \begin{blockofcode}{text}
		list recommendations
	\end{blockofcode}

	\begin{infobox}
		\bulb{} \hspace{.6em}\parbox{0.9\textwidth}{%
			\bld{Tip:} For convenience, you can use either \mono{list recommendations} or \mono{list recommendation}.
		}
	\end{infobox}

	\pagebreak
	In this example, ChopChop knows about some \emph{Milk} that is about to expire within the next week (say, for example, it
	is the 4th of November):

	\begin{figure}[!htbp]\centering\ContinuedFloat*
		\img{60mm}{images/list_recommendations_1.png}
		\caption{There's some milk expiring on the 9th of November}
	\end{figure}

	Then, in addition to showing recipes using in-stock ingredients, it also highlights the \emph{Pancakes} recipe,
	which uses milk:

	\begin{figure}[!htbp]\centering\ContinuedFloat
		\img{145mm}{images/list_recommendations_2.png}
		\caption{The recommendations view}
	\end{figure}

% end subsubsection
% end subsection
