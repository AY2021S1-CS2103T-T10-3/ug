% Part 3.3 - Ingredient Commands.tex

\pagebreak
\hypertarget{GroupIngredientCommands}{}
\subsection{Ingredient Commands}

These commands allow you to add, delete, and edit ingredients.

\hypertarget{ListIngredientCommand}{}
\subsubsection{Listing Ingredients — \texttt{\bld{list} ingredients}}

	This command shows a list of all recipes in ChopChop. As with the \mono{list recipes} command, you can use this command to switch
	between panes without clicking, or to reset any filters.

	\bld{Usage:} \begin{blockofcode}{text}
		list ingredients
	\end{blockofcode}

	\begin{infobox}
		\bulb{} \hspace{.6em}\parbox{0.9\textwidth}{%
			\bld{Tip:} For convenience, you can use either \mono{list ingredients} or \mono{list ingredient}.
		}
	\end{infobox}

	Executing this command simply brings you back to the ingredient list view:

	\begin{figure}[!htbp]\centering
		\img{145mm}{images/list_ingredients.png}
		\caption{The ingredient list view}
	\end{figure}

% end subsubsection


\pagebreak
\hypertarget{AddIngredientCommand}{}
\subsubsection{Adding Ingredients — \texttt{\bld{add} ingredient}}

	This command adds an ingredient to ChopChop, with an optional quantity and expiry date:
	\begin{bulletlist}
		& If the quantity is not specified, ChopChop will infer \bld{1} of a counted quantity, like eggs.
		& If the expiry date is not specified, it is assumed that the ingredient (eg. salt) does not expire.
	\end{bulletlist}

	As mentioned in the \hyperlink{OverviewIngredients}{overview above}, an ingredient can consist of multiple sets;
	the \mono{add ingredient} command will intelligently \emph{combine} ingredients as appropriate.

	\begin{infobox}
		\info{}\hspace{.6em}\parbox{0.9\textwidth}{%
			\bld{Note:} Ingredients need to have compatible units in order to be combined;
			see \hyperlink{QuantitiesAndUnits}{this section} for how it works.
		}
	\end{infobox}

	If the new ingredient has \mono{/tag} options that are not present in the existing ingredient, then they are added as well.

	\bld{Usage:} \begin{blockofcode}{text}
		add ingredient <name>
			[/qty <quantity>]
			[/expiry <expiry-date>]
			[/tag <tag-name>]...
	\end{blockofcode}

	Examples:
	\begin{bulletlist}
		& \mono{add ingredient milk /qty 1l /expiry 2020-11-09} \\
		  This adds one litre of milk that expires on the 9th of November.

		& \mono{add ingredient egg /expiry 2020-12-25} \\
		  This adds \bld{one} egg that expires on Christmas day.
	\end{bulletlist}

	\pagebreak
	Suppose you just finished a grocery run, and want to add the items to ChopChop. First, you have 2 cartons of milk:
	\vspace{-.5em} % asdf

	\begin{figure}[!htbp]\centering\ContinuedFloat*
		\img{145mm}{images/add_ingredient_1.png}
		\caption{Adding 2 litres of milk}
	\end{figure}
	\vspace{-1.5em} % asdf
	Since ChopChop did not know about \enquote{milk} previously, a new ingredient entry is created for it:

	\begin{figure}[!htbp]\centering\ContinuedFloat
		\img{145mm}{images/add_ingredient_2.png}
		\caption{The newly added milk ingredient}
	\end{figure}

	\pagebreak
	Next, suppose you also bought 24 blueberries:

	\begin{figure}[!htbp]\centering\ContinuedFloat
		\img{145mm}{images/add_ingredient_3.png}
		\caption{Adding 24 blueberries}
	\end{figure}
	\vspace{-1.5em} % asdf

	This time, since ChopChop already knew about blueberries, our previous 19 blueberries now become 43:

	\begin{figure}[!htbp]\centering\ContinuedFloat
		\img{145mm}{images/add_ingredient_4.png}
		\caption{You now have 43 blueberries}
	\end{figure}

	If you try to add an ingredient with incompatible quantities (for example, suppose you did not want to count the blueberries
	individually, and you only know that you bought a 400 gram box), ChopChop will display an error message,
	and not update the ingredient:

	\begin{figure}[!htbp]\centering\ContinuedFloat
		\img{75mm}{images/add_ingredient_5.png}
		\caption{Ingredients must have compatible units to be combined}
	\end{figure}

% end subsubsection











\hypertarget{DeleteIngredientCommand}{}
\subsubsection{Deleting Ingredients — \texttt{\bld{delete} ingredient}}

	This command deletes a specific ingredient from ChopChop. Similar to the \mono{add ingredient} command, this command also allows
	you to delete quantities of ingredients instead of the whole ingredient. In this scenario, ChopChop will intelligently remove the
	earliest-expiring ingredients first.

	If \mono{/qty} is not specified, then the behaviour of this command is to completely remove the ingredient from ChopChop.
	Worry not: if you accidentally delete something, you can always \hyperlink{UndoCommand}{\mono{undo}} it.

	\begin{infobox}
		\info{}\hspace{.6em}\parbox{0.9\textwidth}{%
			\bld{Note:} If specified, the quantity needs to have compatible units with the existing ingredient;
			see \hyperlink{QuantitiesAndUnits}{this section} for how it works.
		}
	\end{infobox}

	\bld{Usage:} \begin{blockofcode}{text}
		delete ingredient |\itemref{}|
			[/qty <quantity>]
	\end{blockofcode}

	Examples:
	\begin{bulletlist}
		& \mono{delete ingredient \#4} \\
		  This deletes the fourth ingredient currently shown in the recipe list.

		& \mono{delete ingredient milk /qty 500ml} \\
		  This removes \SI{500}{\milli\litre} of milk from ChopChop's inventory.
	\end{bulletlist}

	\pagebreak
	To illustrate, suppose that you poured yourself a glass of cold milk to drink, without making a recipe. To tell ChopChop that there is less milk in the fridge, you would use this command:
	\vspace{-.5em} % asdf
	\begin{figure}[!htbp]\centering\ContinuedFloat*
		\img{145mm}{images/delete_ingredient_1.png}
		\caption{Removing 350ml of milk}
	\end{figure}
	\vspace{-2em} % asdf

	Notice how the amount of milk decreased from \SI{2}{\litre} to \SI{1.65}{\litre}:
	\vspace{-.5em}
	\begin{figure}[!htbp]\centering\ContinuedFloat
		\img{145mm}{images/delete_ingredient_2.png}
		\caption{You now only have 1.65 litres of milk left}
	\end{figure}
	\vspace{-1.5em}

% end subsubsection











\hypertarget{FindIngredientCommand}{}
\subsubsection{Finding Ingredients — \texttt{\bld{find} ingredient}}

	This command finds all ingredients containing the given keywords in the name, and it works identically to the
	\hyperlink{FindRecipeCommand}{\mono{find recipe}} command above.

	Constraints:
	\begin{romanlist}
		& At least one search keyword must be given
	\end{romanlist}

	\bld{Usage:} \begin{blockofcode}{text}
		find ingredient
			<keyword>
			[<keyword>]...
	\end{blockofcode}

	For example, suppose you wanted to find all ingredients containing fish (not in the literal sense, but only in their name):

	\begin{figure}[!htbp]\centering\ContinuedFloat*
		\img{145mm}{images/find_ingredient_1.png}
		\caption{The complete ingredient list}
	\end{figure}

	\pagebreak
	Now, only the matching ingredients are shown:

	\begin{figure}[!htbp]\centering\ContinuedFloat
		\img{145mm}{images/find_ingredient_2.png}
		\caption{Only ingredients containing \enquote{fish} in their name are shown}
	\end{figure}

	To clear the search filter, you can either click the \emph{Ingredients} button, or use \mono{list ingredients} to return
	to the list of ingredients.

% end subsubsection





\hypertarget{FilterIngredientCommand}{}
\subsubsection{Filtering Ingredients — \texttt{\bld{filter} ingredient}}

	This command filters all ingredients and lists those matching the name keywords, tags, and expiry dates specified in the command.
	Except for the changes in the search fields, this feature works identically to the
	\hyperlink{FilterRecipeCommand}{\mono{filter recipe}} command above. Notably, \bld{all} the search terms have to match an
	ingredient before it is found.

	\bld{Usage:} \begin{blockofcode}{text}
		filter ingredient
			[/name <name-keywords>...]...
			[/tag <tag-keywords>...]...
			[/expiry <expiry-date>]
	\end{blockofcode}

	\begin{bulletlist}
		& Keywords following \mono{/tag} and \mono{/name} do not have to be complete to match the tag or the ingredient's name.
		& Expiry dates (\mono{/expiry <expiry-date>}) filter only ingredients that expire before the date provided, and if there are
			multiple expiry dates specified, only the earliest one will be considered.
	\end{bulletlist}

	Constraints:
	\begin{romanlist}
		& At least one search term must be given, and they should be either \mono{/name}, \mono{/expiry} or \mono{/tag}.
		& Search terms must not be empty.
	\end{romanlist}

	Examples:
	\begin{bulletlist}
		& \mono{filter ingredient /name dark chocolate} \\
		  This matches \emph{dark chocolate} and \emph{dark chocolate syrup}, assuming they are the only ingredients whose
		  names contain \emph{dark chocolate}.

		& \mono{filter ingredient /name dark chocolate /name syrup} \\
		  This matches \emph{dark chocolate syrup}, assuming it is the only ingredient whose name contains \bld{both}
		  \emph{dark chocolate} and \emph{syrup}.

		& \mono{filter ingredient /tag bitter taste} \\
		  This matches \emph{bitter melon} and \emph{dark chocolate}, assuming they are the only ingredients tagged
		  with \emph{bitter taste}.

		& \mono{filter ingredient /tag frequently used /tag sweet} \\
		  This matches \emph{sugar}, if it is the only ingredient tagged \bld{both} \emph{frequently used} and \emph{sweet}.

		& \mono{filter ingredient /expiry 2020-12-01 /expiry 2020-10-31 /expiry 2023-01-01} \\
		  This matches \emph{apple}, if it is the only ingredient expiring before \emph{2020-10-31}. Note that only the
		  earliest date is considered here (in this case, 31st October 2020), and the rest are ignored.

		& \mono{filter ingredient /tag powdery /expiry 2020-12-31 /tag bakery /name soda} \\
		  This matches \emph{baking soda}, assuming it is the only ingredient that matches all the specified criteria.
	\end{bulletlist}

	To illustrate, suppose you want to search for ingredients tagged both \emph{bakery} and \emph{sweet}, expiring before
	\emph{2021-12-31} and containing \emph{sugar} in the name; you can use
	\mono{filter ingredient /tag bakery}\\
	\mono{/expiry 2021-12-31 /tag sweet /name sugar}:

	\begin{figure}[!htbp]\centering\ContinuedFloat*
		\img{145mm}{images/filter_ingredient_1.png}
		\caption{The initial list of ingredients}
	\end{figure}

	After executing the command, similar to the effect of \mono{filter recipe} command, the ingredient list has changed, showing the only matching ingredients, \emph{Brown Sugar} and \emph{Granulated Sugar}:

	\begin{figure}[!htbp]\centering\ContinuedFloat
		\img{145mm}{images/filter_ingredient_2.png}
		\caption{The ingredients matching all the provided criteria}
	\end{figure}

	Note that \emph{Honey}, which was also tagged \emph{bakery} and \emph{sweet} and would expire before \emph{2021-12-31}, was not included because its name does not contain \emph{sugar}.

	To reset the search filter or go back to the full ingredient view, you can click the Ingredients button or run the
	\mono{list ingredients} command.

% end subsubsection






\pagebreak
\hypertarget{EditIngredientCommand}{}
\subsubsection{Editing Ingredients — \texttt{\bld{edit} ingredient}}

	This command edits the given ingredient, in a similar fashion to the \hyperlink{EditRecipeCommand}{\mono{edit recipe}} command.
	However, currently its functionality is limited to only editing the tags of an ingredient (more features coming soon!).

	Do refer to the documention on the \hyperlink{EditRecipeCommand}{\mono{edit recipe}} command above to find out how edit-descriptors
	work; this command currently only supports \mono{/tag:add} and \mono{/tag:delete}.

	It is an error to delete a tag from an ingredient that did not contain that tag, and similarly to add a duplicate tag to an ingredient.

	\bld{Usage:} \begin{blockofcode}{text}
		edit ingredient |\itemref{}|
			[/tag:<action> <tag-name>]...
	\end{blockofcode}

	Examples:
	\begin{bulletlist}
		& \mono{edit ingredient \#4 /tag:add frozen} \\
		  This tags the fourth ingredient currently shown in the ingerdient list with \emph{frozen}.

		& \mono{edit ingredient sprinkles /tag:delete fridge} \\
		  This removes the tag \emph{fridge} from the ingredient named \enquote{sprinkles}.
	\end{bulletlist}

% end subsubsection











% end subsection
